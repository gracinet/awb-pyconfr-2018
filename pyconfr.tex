  \documentclass{beamer}
  \usepackage[utf8]{inputenc}
  \usetheme{Warsaw}
  \usepackage{minted}

  \title{AnyBlok / WMS Base}
  \author{Georges Racinet}
  \institute{Anybox, \url{https://anybox.fr}}

  \pgfdeclareimage[height=10mm,width=10mm]{anyblok-logo}{anyblok-logo_w_256.png}
  \logo{\pgfuseimage{anyblok-logo}}
  \date{pycon.fr 2018}
  \begin{document}

  \begin{frame}
    \titlepage
    \begin{center}
      Un moteur d'applications de logistique avec Python 3, SQLAlchemy,
      PostgreSQL et AnyBlok.
    \end{center}

    Présentation disponible en ligne sur
    \url{https://slides.racinet.fr/2018/pycon.fr.pdf}
  \end{frame}

  \begin{frame}{Anyblok / WMS Base}{Ressources}

  \begin{itemize}
    \item Cette présentation:
      \begin{itemize}
        \item PDF: \url{https://slides.racinet.fr/2018/pycon.fr.pdf}
        \item source:
          \url{https://github.com/gracinet/awb-pyconfr-2018/pyconfr.tex}
      \end{itemize}
    \item Ressources sur AWB:
      \begin{itemize}
        \item code source: \url{https://github.com/AnyBlok/anyblok_wms_base}
        \item nouvelles sur mon blog (en français):
          \url{https://blog.racinet.fr/tag/logistique.html}
        \item documentation de référence (en anglais)
          \url{https://anyblok-wms-base.readthedocs.io/en/latest}
      \end{itemize}
  \end{itemize}
\end{frame}

\begin{frame}{Georges Racinet}
  \begin{itemize}
    \item développeur python pro depuis 2005
    \item fondation Anybox avec Christophe Combelles 2010
      \begin{itemize}
      \item applis gestion entreprise (ERP), en particulier stocks
      \item grande expérience logistique chez Anyboxiens
  \end{itemize}
\end{itemize}
\end{frame}

\begin{frame}{Cas d'utilisation}
  \begin{itemize}
    \item commerce en ligne et en magasins
    \item logistique pure
    \item gestion de matériel
    \item gestion de fabrication
  \end{itemize}
\end{frame}

\begin{frame}{État du projet}
  \begin{itemize}
  \item pure bibliothèque
  \item 100\% libre et développé publiquement
  \item couverture de tests à 100\% depuis le début
  \item documentation de référence (en anglais) exhaustive
  \item version actuelle : 0.8
  \end{itemize}
\end{frame}

\begin{frame}{But de la présentation}
  J'assume…
  \begin{itemize}
    \item<2-> développeurs, développeurs !
      \begin{itemize}
        \item<3-> primo utilisateurs
        \item<4-> contributeurs
        \item<5-> nouvelles briques intermédiaires ?
      \end{itemize}
    \item<6-> nouveau nom ?
  \end{itemize}
\end{frame}

\begin{frame}{Cas d'utilisation}
  Revenons sur les cas d'utilisation…
  \begin{itemize}
    \item commerce en ligne et en magasins
    \item logistique pure
    \item gestion de matériel
    \item gestion de fabrication
  \end{itemize}
\end{frame}

\begin{frame}{Points communs}{Cas d'utilisation}
  \begin{itemize}
    \item<2-> objets physiques
    \item<3-> où ? quand ? comment / pourquoi ?
    \item<4-> prévision / planification
    \item<5-> la réalité est têtue, et elle a le dernier mot !
  \end{itemize}
\end{frame}

\begin{frame}{Motivation et objectifs}
  \begin{itemize}
    \item<2-> minimalisme
    \item<3-> généricité
    \item<4-> liberté
      \begin{itemize}
        \item au sens du logiciel libre
        \item guider le code applicatif sans l'entraver
        \item la mienne
      \end{itemize}
    \item<5-> performance
    \item<6-> qualité
    \item<7-> réconciliation avec la réalité
  \end{itemize}
\end{frame}

\begin{frame}
  {Le scénario}{Un fil conducteur pour les exemples}
  \begin{itemize}
    \item vente de livres en gros 1/2 gros et détail
    \item on va regarder le cas de *A Dance of Dust and Wind* (par Georges L.P.
  Racinet)
    \item 3 volumes
    \item coffret de l'intégrale, préparé à l'entrepôt
  \end{itemize}
\end{frame}

\begin{frame}[fragile]{Objets physiques : le modèle PhysObj}

On récupère un type, puis les objets physiques de ce type :
\begin{minted}[fontsize=\footnotesize,xleftmargin=-4em]{python3}
     >>> PhysObj = registry.Wms.PhysObj
     >>> livre_type = (PhysObj.Type.query()
     ...               .filter_by(code='GR-DUST-WIND-VOL2').one())
     >>> exemplaires = PhysObj.query().filter_by(type=livre_type).all()
     >>> exemplaires
     [Wms.PhysObj(id=18, type=Wms.PhysObj.Type(id=7, code='GR-DUST-WIND-VOL2')),
     Wms.PhysObj(id=19, type=Wms.PhysObj.Type(id=7, code='GR-DUST-WIND-VOL2')),
     Wms.PhysObj(id=20, type=Wms.PhysObj.Type(id=7, code='GR-DUST-WIND-VOL2')),
     Wms.PhysObj(id=21, type=Wms.PhysObj.Type(id=7, code='GR-DUST-WIND-VOL2')),
     Wms.PhysObj(id=22, type=Wms.PhysObj.Type(id=7, code='GR-DUST-WIND-VOL2'))]
\end{minted}

\uncover<2->{Premières remarques:}
\begin{itemize}
  \item<2-> Une ligne par objet physique
  \item<3-> Pas de champ quantité
  \item<4-> Pas de système d'unités, l'unité fait partie du type
\end{itemize}
\end{frame}


\end{document}
